%% IEEE template
%%
%%   A simple, one file template for writing papers for IEEE, based on
%%   the work done by Michael Shell.
%%
%% Version:
%%   v0.0
%%
%% Authors:
%%   Hidde-Jan Jongsma
%%   Michael Shell


% The default document class. Uses a4paper for European users.
\documentclass[final, a4paper]{IEEEtran}

% When working on a technical note, use the following settings.
%\documentclass[final, a4paper, technote]{IEEEtran}

% For drafts, the following documentclass is recommended.
%\documentclass[draftclass, a4paper, 10pt, onecolumn]{IEEEtran}


% *** REFERENCE PACKAGES ***
%
\usepackage{cite}
\usepackage{url}

% The hyperref package creates links between references in the paper.
% You should probably disable it before submitting the paper.
\usepackage{hyperref}


% *** GRAPHICS PACKAGES ***
%
\usepackage[pdftex]{graphicx}

% Uncomment the following line to use tikz.
%\usepackage{tikz}

% Images can be placed in the subdirectory ./img.
\graphicspath{{./img/}}

% The following extensions can be omitted.
\DeclareGraphicsExtensions{.pdf,.jpeg,.jpg,.png}

% If you need to use color, uncomment the next line.
%\usepackage[usenames]{xcolor}

% To automatically convert eps files to pdf files, uncomment the
% following line:
%\usepackage{epstopdf}


% *** MATH PACKAGES ***
%
\usepackage[cmex10]{amsmath}  % Prevents bit-mapped fonts.
\usepackage{amssymb}
\usepackage{amsthm}
\usepackage{array}
\usepackage{mathrsfs}  % Enables \mathscr font.

% Uncomment the following line to allow page breaks in equations.
%\interdisplaylinepenalty=2500

% Correct bad hyphenation here
\hyphenation{op-tical net-works semi-conduc-tor}


\begin{document}

% *** AUTHOR/TITLE SETUP ***
%
% Set the title for the paper.
\title{Paper title}

% Authors of the paper. Do not put any extra spaces between the
% brackets, which includes newlines.
\author{A.~Uthor,
        S.E.~Condauthor,~\IEEEmembership{Member,~IEEE,}
        and A.N.~Otherauthor,~\IEEEmembership{Member,~IEEE}% <-this % stops a space
\thanks{All authors are stupid.}% <-this % stops a space
}

% The paper headers. Set these for peerreview stuff.
\ifCLASSOPTIONpeerreview
\markboth{IEEE Transactions on Automatic Control}%
{Paper title}
\else
\markboth{IEEE Transactions on Automatic Control}%
{Uthor \MakeLowercase{\textit{et al.}}: Paper title}
\fi

% If you want to put a publisher's ID mark on the page you can do it like
% this:
%\IEEEpubid{0000--0000/00\$00.00~\copyright~2012 IEEE}
% Remember, if you use this you must call \IEEEpubidadjcol in the second
% column for its text to clear the IEEEpubid mark.

% For special paper notices, uncomment the following line.
%\IEEEspecialpapernotice{(Invited Paper)}


% Render the title.
\maketitle


% *** ABSTRACT & KEYWORDS ***
%
% As a general rule, do not put math, special symbols or citations
% in the abstract or keywords.
\begin{abstract}
  Short abstract here.
\end{abstract}

% Note that keywords are not normally used for peerreview papers.
\begin{IEEEkeywords}
just,
put,
a few,
here.
\end{IEEEkeywords}

% For peer review papers, you can put extra information on the cover
% page as needed:
% \ifCLASSOPTIONpeerreview
% \begin{center} \bfseries EDICS Category: 3-BBND \end{center}
% \fi
%
% For peerreview papers, this IEEEtran command inserts a page break and
% creates the second title. It will be ignored for other modes.
\IEEEpeerreviewmaketitle


% *** MAIN CONTENT ***
%
\section{Introduction}
% For a full paper, use a IEEEPARstart:
%\IEEEPARstart{T}{his} is the first sentence of the introduction. A
%second sentence will soon follow the first one. Another sentence is
%right out.
%
% For a technote, just start normally.
Here is some text which is part of the introduction, see
\cite{ma10,mesbahi10}.


\subsection{Subsection Heading Here}
Subsection text here.


% needed in second column of first page if using \IEEEpubid
%\IEEEpubidadjcol


% *** APPENDICES ***
%
% If appendices are needed, uncomment the following line and continue
% normally.
%\appendices


% *** ACKNOWLEDGEMENT ***
%
% Use an unmarked section* for acknowledgement.
%\section*{Acknowledgment}


% *** BIBLIOGRAPHY ***
%
% You can trigger a \newpage just before the given reference number,
% used to balance the columns on the last page adjust value as needed,
% may need to be readjusted if the document is modified later.
%\IEEEtriggeratref{8}
% The "triggered" command can be changed if desired:
%\IEEEtriggercmd{\enlargethispage{-5in}}
\bibliographystyle{IEEEtran}
% argument is your BibTeX string definitions and bibliography database(s)
\bibliography{./IEEEabrv,./db}


% *** BIOGRAPHY ***
%
% If you have an EPS/PDF photo (graphicx package needed) extra braces are
% needed around the contents of the optional argument to biography to prevent
% the LaTeX parser from getting confused when it sees the complicated
% \includegraphics command within an optional argument. (You could create
% your own custom macro containing the \includegraphics command to make things
% simpler here.)
%\begin{IEEEbiography}[{\includegraphics[width=1in,height=1.25in,clip,keepaspectratio]{mshell}}]{Michael Shell}


\end{document}

